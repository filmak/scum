This report serves to document the two years of work behind the development of the Single Chip Mote digital system, and pass on the knowledge obtained during this process to those who continue to iterate and improve on this initial design. A tested and functioning FPGA prototype is presented, with a built-in ARM Cortex-M0 microprocessor, radio controller, radio timer, and ADC interface. Instructions on how to install the FPGA toolchain and software development tools are included, as well as overviews of their purpose and use in the Single Chip Mote project. This document also covers the testing procedures used to verify this design, and the changes required to take the FPGA-based design and create an ASIC.

The established architecture and interfaces to the radio and analog circuits are merely the bare minimum required for a fully-functioning Single Chip Mote; this project still has a long way to go before it is ready to interface with embedded sensors and microrobots. In the short-term, this project still requires an interface for hardware debugging, and system-level simulations for its mixed-signal interfaces, before it is ready for tapeout in August 2016. In the long-term, this project requires a dedicated group of hardware and software designers to converge on the preferred system-level specifications for the ideal wireless sensor node and microrobot controller. The Single Chip Mote digital system also lacks power management hardware for powering down modules that are not in use. Once it is completely solar powered, the Single Chip Mote will also need on-chip nonvolatile memory and brownout detection circuitry to operate in environments with inconsistent levels of illumination, and energy storage solutions to continue operating in environments with little or no light. Finally, achieving the optimal design in terms of energy consumption requires design space exploration to find the best combination of voltage and frequency while still meeting the requirements of software developers.

The Single Chip Mote is an incredibly ambitious project. Integrating a fully-functioning microprocessor, radio, and sensors onto a single die with zero external components is unprecedented in both academia and industry. That being said, the Single Chip Mote team is composed of hardworking and resourceful engineers, who will undoubtedly prove that this is both achievable and useful for real-world applications. 